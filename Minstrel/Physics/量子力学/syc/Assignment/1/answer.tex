\documentclass[12pt]{article}
\usepackage{amsmath, amssymb, physics, geometry}
\usepackage{braket}
\geometry{a4paper, margin=1in}
\usepackage{hyperref}

\title{The Answer of Assignment 1}
\author{}
\date{}

\begin{document}

\maketitle

\begin{center}
    \textbf{Problem 1 Solution}
\end{center}

\textbf{(1)} In the position representation, from the eigenvalue equation $\hat{H}\psi(x) = h_n \psi(x)$ and the Hamiltonian operator $\hat{H} = -\frac{\hbar^2}{2m} \nabla^2$, we have:
\begin{align*}
    -\frac{\hbar^2}{2m} \nabla^2 \psi(x) = h_n \psi(x)
\end{align*}
The general solution is:
\begin{align*}
    \psi(x) = c_1 e^{w_1 x} + c_2 e^{w_2 x}, \quad w_1 = i\sqrt{\frac{2m h_n}{\hbar^2}}, \quad w_2 = -i\sqrt{\frac{2m h_n}{\hbar^2}}
\end{align*}
Given $p_n = \frac{2\pi\hbar n}{L}$, $h_n = \frac{2\pi^2 \hbar^2 n^2}{L m}$. Take $\psi_n(x) = e^{\frac{i p_n x}{\hbar}}$ as an example, the eigenvalue corresponding to $\ket{\psi_n}$ is $\frac{2\pi^2 \hbar^2 n^2}{L m}$.

\vspace{1em}
\textbf{(2)} 
\begin{align*}
    \braket{\psi_{n_1} | \psi_{n_2}} &= \int_{-\infty}^{+\infty} \psi_{n_1}^*(x)\, \psi_{n_2}(x) \, dx \\
    &= \int_{-\infty}^{+\infty} e^{-\frac{i p_{n_1} x}{\hbar}} e^{\frac{i p_{n_2} x}{\hbar}} dx \\
    &= \int_{-\infty}^{+\infty} e^{\frac{i (p_{n_2} - p_{n_1}) x}{\hbar}} dx \\
    &= \int_{-\infty}^{+\infty} e^{\frac{i 2\pi (n_2 - n_1)x}{L}} dx \\
    &= \lim_{l \to +\infty} \int_{-l}^{+l} e^{\frac{i 2\pi (n_2 - n_1)x}{L}} dx \\
    &= \lim_{l \to +\infty} \frac{L \sin\left[\frac{2\pi (n_2 - n_1) l}{L}\right]}{\pi (n_2 - n_1)}
\end{align*}
When $n_2 = n_1$, $\braket{\psi_{n_1} | \psi_{n_2}} \to \infty$ and
\begin{align*}
    \lim_{l \to +\infty} \int_{-\infty}^{+\infty} \frac{\sin(lx)}{x} dx = \pi
\end{align*}
So,
\begin{align*}
    \braket{\psi_{n_1} | \psi_{n_2}} = L \delta_{n_1 n_2}
\end{align*}
\textbf{(3)} 
When $L \to \infty$, $p_n$ becomes continuous and $\psi_n(x)$ becomes a plane wave.
\[
\braket{\psi_{p_1} | \psi_{p_2}} = \int_{-\infty}^{+\infty} e^{-\frac{i p_1 x}{\hbar}} e^{\frac{i p_2 x}{\hbar}} dx = \lim_{l \to +\infty} \frac{2\hbar \sin\left[\frac{(p_2 - p_1) l}{\hbar}\right]}{ (p_2 - p_1)} =  2\pi\delta(\frac{p_2 - p_1}{\hbar})=2\pi\hbar\delta(p_2-p_1)
\]
\vspace{1em}
\textbf{(4)} 
In the position representation, $\ket{\psi_{x_0,\epsilon}}=\int_{-\infty}^{+\infty}\psi_{x_0,\epsilon}(x)\ket{x}dx $ $$\braket{\delta | \delta}=\int_{-\infty}^{+\infty}(x\psi_{x_0,\epsilon}(x)-x_0 \psi_{x_0,\epsilon}(x))^{*}(x\psi_{x_0,\epsilon}(x)-x_0 \psi_{x_0,\epsilon}(x))dx$$
Considering the Gaussian wave packet $\psi_{x_0,\epsilon}(x)={(\frac{1}{{2\pi\epsilon^2}})}^{1/4}e^{\frac{-{(x-x_0)}^2}{{4\epsilon}^2}}$
\[
\int_{-\infty}^{+\infty}\psi^*\psi dx=\int_{-\infty}^{+\infty}\frac{1}{\sqrt{2\pi\epsilon^2}}e^{-\frac{(x-x_0)^2}{2\epsilon^2}}dx=\int_{-\infty}^{+\infty}\frac{1}{\sqrt{2\pi}}e^{\frac{-y^2}{2}}dy=1
\]
and 
\[
\int_{-\infty}^{+\infty}(x-x_0)^2\frac{1}{\sqrt{2\pi\epsilon^2}}e^{-\frac{(x-x_0)^2}{2\epsilon^2}}dx
= \left. -\frac{\partial}{\partial a} \int_{-\infty}^{+\infty} \frac{1}{\sqrt{2\pi\epsilon^2}} e^{-a(x-x_0)^2} dx \right|_{a = \frac{1}{2\epsilon^2}}
\]
Note that
\[
\int_{-\infty}^{+\infty} e^{-a(x-x_0)^2} dx = \sqrt{\frac{\pi}{a}}
\]
Therefore,
\[
-\frac{\partial}{\partial a} \sqrt{\frac{\pi}{a}} = \frac{1}{2} \sqrt{\pi} a^{-3/2}
\]
Substituting $a = \frac{1}{2\epsilon^2}$, we get
\[
\int_{-\infty}^{+\infty}(x-x_0)^2\frac{1}{\sqrt{2\pi\epsilon^2}}e^{-\frac{(x-x_0)^2}{2\epsilon^2}}dx = \epsilon^2
\]
To ensure $\braket{\delta | \delta} < \epsilon$, we can set $\delta = \frac{\sqrt{\epsilon}}{2}$, so $\psi_{x_0,\delta}(x) = \frac{1}{\sqrt[4]{2\pi\delta^2}} e^{-\frac{(x-x_0)^2}{4\delta^2}}$. According to the previous derivation,
\[
\braket{\delta | \delta} = \int_{-\infty}^{+\infty} \left( x \psi_{x_0,\delta}(x) - x_0 \psi_{x_0,\delta}(x) \right)^* \left( x \psi_{x_0,\delta}(x) - x_0 \psi_{x_0,\delta}(x) \right) dx = \delta^2 = \frac{\epsilon}{4}
\]
this satisfies the condition $\braket{\delta | \delta} < \epsilon$.

\begin{center}
    \textbf{Problem 2 Solution}
\end{center}
\textbf{(1)} Consider a qubit with Hamiltonian $$H = {-}\mu BY$$ and and the initial state at time t = 0 is $\ket{\uparrow}$.
We have\[ \hat{H}\ket{\psi}=i\hbar \frac{d}{dt}\ket{\psi} \]  
The time evolution of the state is given by:
\begin{align*}
    \ket{\psi(t)} &= e^{-\frac{i}{\hbar} H t} \ket{\uparrow} \\
    &= e^{-\frac{i}{\hbar} (-\mu B Y) t} \ket{\uparrow} \\
    &= e^{i \frac{\mu B}{\hbar} Y t} \ket{\uparrow}
\end{align*}
Using Taylor expansion, we can express this as:
\[\ket{\psi(t)} = \ket{\uparrow} \left( 1 + i \frac{\mu B}{\hbar} Y t - \frac{1}{2} \left( \frac{\mu B}{\hbar} Y t \right)^2 + \cdots \right)=\cos(\theta)\ket{\uparrow}{-}\sin(\theta)\ket{\downarrow}\]
where $\theta = \frac{\mu B}{\hbar} t$. The probability of measuring $\ket{\uparrow}$ at time $t$ is $p_{\uparrow}(t) = \cos^2(\theta)$, and the probability of measuring $\ket{\downarrow}$ is $p_{\downarrow}(t) = \sin^2(\theta)$.
The expectation value ⟨Z⟩ measured at this time is given by:
\begin{align*}
    \braket{Z} &= \braket{\psi(t) | Z | \psi(t)} = \cos^2(\theta) \braket{\uparrow | Z | \uparrow} + \sin^2(\theta) \braket{\downarrow | Z | \downarrow} + 2\cos(\theta)\sin(\theta) \braket{\uparrow | Z | \downarrow}
\end{align*}
where $\braket{\uparrow | Z | \uparrow} = 1$, $\braket{\downarrow | Z | \downarrow} = -1$, and $\braket{\uparrow | Z | \downarrow} = 0$. Thus, we have:
\begin{align*}
    \braket{Z} &= \cos^2(\theta) - \sin^2(\theta) \\
    &= \cos(2\theta)
\end{align*}

\textbf{(2)} 
We can record the n-th measurement result as $S_{n}(0/1)$ and we have:

\[
\left\{
\begin{aligned}
    &p(S_{0}(0)) = 1 \\
    &p(S_{1}(0)) = \cos^2(\theta) \cdot 1 \\
    &\cdots \\
    &p(S_{n}(0)) = \cos^2(\theta) \cdot p(S_{n-1}(0)) + \sin^2(\theta) \cdot p(S_{n-1}(1)) \\
\end{aligned}
\right.
\]
\[
\left\{
\begin{aligned}
    &p(S_{0}(1)) = 0 \\
    &p(S_{1}(1)) = \sin^2(\theta) \cdot 1 \\
    &\cdots \\
    &p(S_{n}(1)) = \sin^2(\theta) \cdot p(S_{n-1}(0))+ \cos^2(\theta) \cdot p(S_{n-1}(1)) \\
\end{aligned}
\right.
\]

The number of all the possible sequences at the length $n$ is $2^n$. We use the method of induction to testify the sum of the probabilities of all sequences at the length $n$ is 1.

\textbf{Base case:} For $n=1$, we have:
\[
p(S_{1}(0)) + p(S_{1}(1)) = \cos^2(\theta) + \sin^2(\theta) = 1
\]

\textbf{Inductive step:} Assume it holds for $n=k$, i.e.,
\[
\sum_{i=0}^{2^k} p_i = 1.
\]

Now consider $n = k+1$:
\[
\sum_{i=0}^{2^{k+1}} p_i
= \sum_{i=0}^{2^k} p_i \cdot \bigl(\cos^2(\theta) + \sin^2(\theta)\bigr)
= \sum_{i=0}^{2^k} p_i \cdot 1
= \sum_{i=0}^{2^k} p_i
= 1.
\]

\textbf{(3)} When $\delta t\to 0$ while T is fixed:
\[
p(\ket{00\cdots 0}) = \lim_{\delta t \to 0} p(S_{n}(0)) = \lim_{\delta t \to 0} \cos^{2[\frac{T}{\delta t}]}(\frac{\mu B \delta t}{\hbar}) = e^{\frac{2T}{\delta t}\ln(cos(\frac{\mu B \delta t}{\hbar}))}=1
\]

\textbf{(4)}  At $t = T$ , what is the probability to find the qubit in state$\ket{\uparrow}$, and what is the probability to find the qubit in state$\ket{ \downarrow⟩}$? What is the expectation value$ \braket{Z}$ measured at this time? 
we have:
\[
p_{n+1}(0)=p_{n}(0)\cdot\cos^2(\theta)+(1-p_{n}(0))\cdot \sin^2(\theta) 
\]
\[
p_{n+1}(0)-\frac{1}{2}=\cos(2\theta)(p_{n}(0)-\frac{1}{2})
\]
as a result, we have:
$$p_n(0)=\lim_{n\to \infty}cos^{\frac{T}{\delta t}}(2\theta)\cdot\frac{1}{2}+\frac{1}{2}=1$$
\[
p_n(1)=\frac{1}{2}-cos^{\frac{T}{\delta t}}(2\theta)\cdot\frac{1}{2}
\]
the expectation value $\braket{Z}$ measured at this time is given by:
\[
\braket{Z}=\cos^{\frac{T}{\delta t}}(2\theta)\]

\begin{center}
    \textbf{Problem 3 Solution}
\end{center}

\textbf{(1)} 
A self-adjoint operator $O$ in Hilbert space can be spectrally decomposed. Suppose its eigenvalues are $\lambda_n$ and the corresponding eigenstates are $\ket{\phi_n}$, then:
\[
\braket{u|O|v} = \sum_n \lambda_n \braket{u|\phi_n}\braket{\phi_n|v} \le \max_n |\lambda_n| \sqrt{\braket{u|u} \braket{v|v}} = \max_n |\lambda_n| \|u\| \|v\| 
 \]

\textbf{(2)}
We try to prove
\[
\ket{\psi_2} = \cos(\theta) e^{i\alpha} \ket{\psi_1} + \sin(\theta) e^{i\beta} \ket{\psi_w}
\]
 In linear algebra, an inner product space can be regarded as the direct sum of a subspace and its orthogonal complement. For $\ket{\psi_2}$, it can be decomposed as $\ket{\psi_2} = \cos(\theta) e^{i\alpha} \ket{\psi_1} + \sin(\theta) e^{i\beta} \ket{\psi_w}$, where $\ket{\psi_1}$ is the projection of $\ket{\psi_2}$ onto $\ket{\psi_1}$, and $\ket{\psi_w}$ is the projection of $\ket{\psi_2}$ onto $\ket{\psi_w}$. Since $\ket{\psi_1}$ and $\ket{\psi_w}$ are orthogonal, their inner product is 0, i.e., $\braket{\psi_1 | \psi_w} = 0$.

\textbf{(3)}
\[
|\braket{\psi_1|O|\psi_1}-\braket{\psi_2|O|\psi_2}|=|(1-\cos^2(\theta))\braket{\psi_1|O|\psi_1} - cos(\beta-\alpha)sin(2 \theta)\braket{\psi_1|O|\psi_w}|
\]
\textbf{(4)}
\[
\left\{
\begin{aligned}
    &1-|\braket{\psi_1|\psi_2}|\le \epsilon \\
    &\braket{\psi_1|O|\psi_2} \le \lambda\\
\end{aligned}
\right.
\]

$|\braket{\psi_1|O|\psi_1}-\braket{\psi_2|O|\psi_2}|=|\braket{\psi_1|O|\psi_1}-\braket{\psi_1|O|\psi_2}+\braket{\psi_1|O|\psi_2}-\braket{\psi_2|O|\psi_2}|= \\
|\braket{\psi_1|O|\ket{\psi_1}-\ket{\psi_2}}+\braket{\bra{\psi_1}-\bra{\psi_2}|O|\psi_2}|\le ||\psi_1||\cdot||O||\cdot||\psi_1-\psi_2||+||\psi_2||\cdot||O||\cdot||\psi_1-\psi_2|| \\
\le 2|\lambda|\cdot \sqrt{{(1+cos(\theta))}^2}=2|\lambda|\cdot \sqrt{2\epsilon}$
\end{document}
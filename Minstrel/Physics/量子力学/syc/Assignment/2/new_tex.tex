\documentclass[12pt]{article}
\usepackage{amsmath, amssymb, physics, geometry}
\usepackage{braket}
\usepackage{ctex}
\geometry{a4paper, margin=1in}
\usepackage{hyperref}

\title{The Answer of Assignment 2}
\author{WEI SHUANG}
\date{31/8/2025}

\begin{document}
\maketitle

\section*{Problem 1 Solution}

\subsection*{(1)}
According to the theory of isomorphism of linear spaces, any two linear spaces of the same dimension are isomorphic. Therefore, 
\[
\sum_{i,j=0}^{1}c_{ij}\ket{i,j}
\]
is isomorphic to $\{(x_1,x_2,x_3,x_4)\}, x\in \mathbb{C}$. That is,
\[
v_1=(c_{00},c_{01},c_{10},c_{11})^T,\quad v_2=(d_{00},d_{01},d_{10},d_{11})^T.
\]

\subsection*{(2)}
We have
\[
\braket{\psi_1|\psi_2}=v_1^\dagger v_2,
\]
because
\[
\braket{\psi_1|\psi_2}=
\left(\sum_{i,j=0}^1c_{ij}^*\bra{ij}\right)
\left(\sum_{i,j=0}^{1}d_{ij}\ket{ij}\right)
= v_1^\dagger v_2.
\]

\subsection*{(3)}
\[
O\ket{pq}=O_{ij,kl}\ket{ij}\bra{kl}\ket{pq}
=O_{ij,kl}\ket{ij}\delta_{kp}\delta_{lq}
=O_{ij,pq}\ket{ij}.
\]
If we define $\ket{00}=e_1$, $\ket{01}=e_2$, $\ket{10}=e_3$, $\ket{11}=e_4$, then we have
\[
Oe_i=\sum_{j=1}^4O_{ji}e_j.
\]
Therefore, the matrix representation of $O$ is:
\[
M=\begin{pmatrix}
O_{11} & O_{12} & O_{13} & O_{14} \\
O_{21} & O_{22} & O_{23} & O_{24} \\
O_{31} & O_{32} & O_{33} & O_{34} \\
O_{41} & O_{42} & O_{43} & O_{44}
\end{pmatrix}.
\]

\subsection*{(4)}
$O\ket{\psi_1}$ is the same as $Mv_1$, because
\[
O\ket{\psi_1}
=O_{ij,kl}\ket{ij}\bra{kl}\cdot c_{mn}\ket{mn}
=O_{ij,mn}c_{mn}\ket{ij}.
\]
This is equivalent to $Mv_1$ if we define $\ket{00}=e_1$, $\ket{01}=e_2$, $\ket{10}=e_3$, $\ket{11}=e_4$.

\section*{Problem 2 Solution}

\subsection*{(1)}
Starting from $[b_i^\dagger,b_j^\dagger]=0$, show that $[b_i,b_j]=0$.  
Indeed,
\[
[b_i^\dagger,b_j^\dagger]=b_i^\dagger b_j^\dagger - b_j^\dagger b_i^\dagger=0.
\]
Taking the Hermitian conjugate,
\[
(b_i^\dagger b_j^\dagger - b_j^\dagger b_i^\dagger)^\dagger = b_j b_i - b_i b_j = 0,
\]
so
\[
[b_i,b_j]=0.
\]

\subsection*{(2)}
\[
\braket{\overline{n_1' n_2' \cdots n_k'}|b_i^\dagger|\overline{n_1 n_2 \cdots n_k}}
=\delta_{n_1,n_1'}\cdots \delta_{n_i,n_i'-1}\cdots \delta_{n_k,n_k'}\sqrt{n_i'}.
\]
Also,
\[
b_i^\dagger \ket{\overline{n_1 n_2 \cdots n_k}} = \sqrt{n_i + 1} \ket{\overline{n_1 n_2 \cdots (n_i + 1) \cdots n_k}},
\]
so
\[
b_i \ket{\overline{n_1 n_2 \cdots n_k}} = \sqrt{n_i} \ket{\overline{n_1 n_2 \cdots (n_i - 1) \cdots n_k}}.
\]

\subsection*{(3)}
\[
\begin{aligned}
[b_i, b_j^\dagger] \ket{\overline{n_1 n_2 \cdots n_k}}
&= b_i b_j^\dagger \ket{\overline{n}} - b_j^\dagger b_i \ket{\overline{n}}.
\end{aligned}
\]

\subsection*{(4)}
If $i = j$, we have
\[
[b_i, b_i^\dagger] \ket{\overline{n}} 
= (n_i + 1 - n_i) \ket{\overline{n}} 
= \ket{\overline{n}}.
\]

\subsection*{(5)}
\[
\{f_i, f_j\} = f_i f_j + f_j f_i.
\]
When $i = j$,
\[
\{f_i, f_i\} = 2 f_i^2 = 0,
\]
since $f_i^2=0$.

\subsection*{(6)}
For $i \neq j$, similarly one checks on the occupation basis that
\[
\{f_i, f_j\} \ket{n}=0.
\]

\subsection*{(7)}
For $i=j$,
\[
\{f_i,f_i^\dagger\}=f_i f_i^\dagger+f_i^\dagger f_i=1.
\]
For $i\neq j$, one finds
\[
\{f_i,f_j^\dagger\}=0.
\]

\section*{Problem 3 Solution}

\subsection*{(1)}
Define creation/annihilation operators on the occupation basis:
\[
b_i^\dagger\ket{n_1,\dots,n_i=0,\dots,n_L}=\ket{n_1,\dots,1,\dots,n_L},\quad
b_i^\dagger\ket{n_1,\dots,n_i=1,\dots,n_L}=0,
\]
\[
b_i\ket{n_1,\dots,n_i=1,\dots,n_L}=\ket{n_1,\dots,0,\dots,n_L},\quad
b_i\ket{n_1,\dots,n_i=0,\dots,n_L}=0.
\]

One checks that
\[
[b_i,b_j]=0,\quad [b_i,b_j^\dagger]=\delta_{ij}.
\]

\subsection*{(2)}
Define
\[
\sigma_i^{\pm}=\frac{1}{2}(\sigma_i^x \pm i\sigma_i^y).
\]
Then
\[
[\sigma_i^-,\sigma_j^-]=0,\quad [\sigma_i^-,\sigma_j^+]=0\ (i\neq j),\quad
\{\sigma_i^-,\sigma_i^+\}=I.
\]

\subsection*{(3)}
Using the Jordan–Wigner transformation:
\[
f_i^\dagger = \left( \prod_{j<i} \sigma_j^z \right) \sigma_i^+,\quad
f_i = \left( \prod_{j<i} \sigma_j^z \right) \sigma_i^-.
\]

Thus
\[
f_i^\dagger f_i = \sigma_i^+ \sigma_i^- = \frac{1}{2}(1 - \sigma_i^z).
\]
Moreover,
\[
\{f_i, f_j\}=0,\quad \{f_i,f_j^\dagger\}=\delta_{ij}.
\]

\subsection*{(4)}
We can invert:
\[
\sigma_i^z = 1 - 2 f_i^\dagger f_i,
\]
\[
\sigma_i^+ = f_i^\dagger \prod_{j<i}(1-2f_j^\dagger f_j),\quad
\sigma_i^- = f_i \prod_{j<i}(1-2f_j^\dagger f_j).
\]

\end{document}

\documentclass[12pt]{article}
\usepackage{amsmath, amssymb, physics, geometry}
\usepackage{braket}
\usepackage{braket}
\usepackage{ctex}
\geometry{a4paper, margin=1in}
\usepackage{hyperref}

\title{The Answer of Assignment 2}
\author{WEI SHUANG}
\date{31/8/2025}

\begin{document}
\maketitle

\begin{center}
    \textbf{Problem 1 Solution}
\end{center}

\textbf{(1)}
According to the theory of isomorphism of linear spaces, any two linear spaces of the same dimension are isomorphic. Therefore, $\sum\limits_{i,j=0}^{1}c_{ij}\ket{i,j}$ is isomorphic to $\{(x_1,x_2,x_3,x_4)\}, x\in \mathbb{C}$. That is, $v_1=(c_{00},c_{01},c_{10},c_{11})^T$, $v_2=(d_{00},d_{01},d_{10},d_{11})^T$.




\textbf{(2)}
$\braket{\psi_1|\psi_2}=v_1^\dagger v_2$成立,这是因为
$\braket{\psi_1|\psi_2}=(\sum_{i,j=0}^1c_{i,j}^*\bra{ij})(\sum{i,,j=0}^1c_{i,j}\ket{ij})=v_1^\dagger v_2$

\textbf{(3)}
$O\ket{pq}=O_{ij,kl}\ket{ij}\bra{kl}\ket{pq}=O_{ij,kl}\ket{ij}\delta_{kp}\delta_{lq}=O_{ij,pq}\ket{ij}$ 
if we define $\ket{00}$ as $e_1$, $\ket{01}$ as $e_2$, $\ket{10}$ as $e_3$, $\ket{11}$ as $e_4$, then we can get the matrix representation of $O$:
$Oe_i=\sum_{j=1}^4O_{ji}e_j$
Therefore, the matrix representation of $O$ is:
\[M=\begin{pmatrix}
O_{11} & O_{12} & O_{13} & O_{14} \\
O_{21} & O_{22} & O_{23} & O_{24} \\
O_{31} & O_{32} & O_{33} & O_{34} \\
O_{41} & O_{42} & O_{43} & O_{44}    
\end{pmatrix}\]

\textbf{(4)}
$O\ket{\psi_1}$ is the same as $Mv_1$,because:

\[
O\ket{\psi_1}=O_{ij,kl}\ket{ij}\bra{kl}\cdot c_{mn}\ket{mn}=O_{ij,kl}\ket{ij}c_{mn}\delta_{km}\delta_{ln}=O_{ij,mn}c_{mn}\ket{ij}
\]
this is equivalent to $Mv_1$ if we define $\ket{00}$ as $e_1$, $\ket{01}$ as $e_2$, $\ket{10}$ as $e_3$, $\ket{11}$ as $e_4$.:

\begin{center}
    \textbf{Problem 2 Solution}
\end{center}

\textbf{(1)}
Starting from $[b_i^\dagger,b_i^\dagger]=0$,show that $[b_i,b_j]=0$.
According to the definition of commutation relation, we have:
\[[b_i^\dagger,b_j^\dagger]=b_i^\dagger b_j^\dagger - b_j^\dagger b_i^\dagger=0\]
Taking the Hermitian conjugate of both sides, we get:
\[(b_i^\dagger b_j^\dagger - b_j^\dagger b_i^\dagger)^\dagger = 0^\dagger\]

\[b_j b_i - b_i b_j = 0\]
so we have:$[b_i,b_j]=0$


\textbf{(2)}

\[
\braket{\overline{n_1' n_2' \cdots n_k'}|b_i^\dagger|\overline{n_1 n_2 \cdots n_k}} = \braket{\overline{n_1 n_2 \cdots n_k}|b_i|\overline{n_1' n_2' \cdots n_k'}}^*=\delta_{n_1,n_1'}\delta_{n_2,n_2'}\cdots \delta_{n_i,n_i'-1}\cdots \delta_{n_k,n_k'}\sqrt{n_i'}
\]
and \[b_i^\dagger \ket{\overline{n_1 n_2 \cdots n_k}} = \sqrt{n_i + 1} \ket{\overline{n_1 n_2 \cdots (n_i + 1) \cdots n_k}}\]
we can get the relation between $b_i$ and $b_i^\dagger$ as follows:
\[
b_i \ket{\overline{n_1 n_2 \cdots n_k}} = \sqrt{n_i} \ket{\overline{n_1 n_2 \cdots (n_i - 1) \cdots n_k}}
\] 

\textbf{(3)}


We have:
\[
\begin{aligned}
[b_i, b_j^\dagger] \ket{\overline{n_1 n_2 \cdots n_k}} &= b_i b_j^\dagger \ket{\overline{n_1 n_2 \cdots n_k}} - b_j^\dagger b_i \ket{\overline{n_1 n_2 \cdots n_k}} \\
&= b_i\sqrt{n_j+1}\ket{\overline{n_1 n_2 \cdots (n_j+1) \cdots n_k}} \\
&\quad - b_j^\dagger\sqrt{n_i}\ket{\overline{n_1 n_2 \cdots (n_i-1) \cdots n_k}} \\
&=\sqrt{n_i(n_j+1)}\ket{\overline{n_1 n_2 \cdots (n_i-1) \cdots (n_j+1) \cdots n_k}}\cdot0 \\
\end{aligned}
\]


\textbf{(4)}
if $i = j$, we have:

\[[b_i, b_i^\dagger] \ket{\overline{n_1 n_2 \cdots n_k}} = (n_i + 1 - n_i) \ket{\overline{n_1 n_2 \cdots n_k}} = \ket{\overline{n_1 n_2 \cdots n_k}}\]

\textbf{(5)}
\[
\{f_i, f_j\} = f_i f_j + f_j f_i
\]
When $i = j$, this becomes
\[
\{f_i, f_i\} = f_i f_i + f_i f_i = 2 f_i^2
\]
Since $f_i$ is a fermionic annihilation operator, it satisfies $f_i^2 = 0$. Therefore,
\[
\{f_i, f_i\} = 2 \times 0 = 0
\]
So, when $i = j$, the anticommutation relation $\{f_i, f_j\} = 0$ holds.

\textbf{(6)}

Suppose $i \neq j$. Calculate $\{f_i, f_j\} \ket{n_1 n_2 \cdots}$. 

Recall the definition:
\[
\{f_i, f_j\} = f_i f_j + f_j f_i
\]
We need to consider the action of $f_i$ and $f_j$ on the occupation number basis $\ket{n_1 n_2 \cdots}$, where $n_k = 0$ or $1$ for fermions.

Consider the following cases:

\textbf{Case 1: $n_i = 0$ or $n_j = 0$}

- If $n_i = 0$, then $f_i \ket{n_1 n_2 \cdots} = 0$.
- If $n_j = 0$, then $f_j \ket{n_1 n_2 \cdots} = 0$.

Therefore, in either case, both $f_i f_j \ket{n_1 n_2 \cdots} = 0$ and $f_j f_i \ket{n_1 n_2 \cdots} = 0$, so
\[
\{f_i, f_j\} \ket{n_1 n_2 \cdots} = 0
\]

\textbf{Case 2: $n_i = 1$ and $n_j = 1$}

- $f_j \ket{n_1 n_2 \cdots n_j = 1 \cdots n_i = 1 \cdots} = (-1)^{s_1} \ket{n_1 \cdots n_j = 0 \cdots n_i = 1 \cdots}$
- Then $f_i$ acts on this state: $f_i \ket{n_1 \cdots n_j = 0 \cdots n_i = 1 \cdots} = (-1)^{s_2} \ket{n_1 \cdots n_j = 0 \cdots n_i = 0 \cdots}$

Similarly, $f_i f_j \ket{n_1 n_2 \cdots}$ and $f_j f_i \ket{n_1 n_2 \cdots}$ will differ by a sign, but since $i \neq j$, the sum $f_i f_j + f_j f_i$ will always cancel out due to the anticommutation property of fermionic operators.

Therefore,
\[
\{f_i, f_j\} \ket{n_1 n_2 \cdots} = 0
\]

\textbf{(7)}
\textbf{case 1: $i=j$}
\[
{f_i,f_j^\dagger} = f_i f_i^\dagger + f_i^\dagger f_i
\]

when $ n_i=0$, we have:
\[
\begin{aligned}
&f_i f_i^\dagger \ket{n_1 n_2 \cdots n_i=0 \cdots}  \\
&=f_i (-1)^s \ket{n_1 n_2 \cdots n_i=1 \cdots}  \\
&=(-1)^s (-1)^s \ket{n_1 n_2 \cdots n_i=0 \cdots} \\
&=\ket{n_1 n_2 \cdots n_i=0 \cdots} \\
\end{aligned}
\]
when $ n_i=1$, we have: 
\[\begin{aligned}
&f_i^\dagger f_i \ket{n_1 n_2 \cdots n_i=1 \cdots} = f_i^\dagger (-1)^s \ket{n_1 n_2 \cdots n_i=0 \cdots} \\
&=(-1)^s (-1)^s \ket{n_1 n_2 \cdots n_i=1 \cdots}=\ket{n_1 n_2 \cdots n_i=1 \cdots}\\
\end{aligned}
\]


\textbf{case 2: $i\neq j$}

\[
\{f_i, f_j^\dagger\} = f_i f_j^\dagger + f_j^\dagger f_i
\]
when $ n_i=0$ or $n_j=0$, we have:
$f_if_j^\dagger\psi=0 f_j^\dagger f_i\psi=0$
when $ n_i=1$ and $n_j=1$, we have the same conclusion as above.

when $n_i=1$ and $n_j=0$, we have:
\[
\begin{aligned}
    &\{f_i f_j^\dagger\} \ket{n_1 n_2 \cdots n_i=1 \cdots n_j=0 \cdots} = f_i (-1)^{\sum_{1}^{j-1}n_x} \ket{n_1 n_2 \cdots n_i=1 \cdots n_j=1 \cdots} \\
    &+ f_j^\dagger (-1)^{\sum_{1}^{i-1}n_x} \ket{n_1 n_2 \cdots n_i=0 \cdots n_j=0 \cdots}=(-1)^{s1+s2}+(-1)^{s1+s2+1}\ket{n_1,\cdots n_i=0 \cdots n_j=1 \cdots} \\
\end{aligned}
\]
\begin{center}
    \textbf{Problem 3 Solution}
\end{center}

\textbf{(1)}
We define $b_I^\dagger$ in 
\[
\begin{aligned}
&b_i^\dagger\ket{n_1,n_2,\cdots,n_k}=\ket{n_1,n_2,\cdots,n_i+1,\cdots,n_L}=\ket{n_1,n_2,\cdots,1,n_{i+1},\cdots,n_L}\text{if }n_i=0 \\
&b_i^\dagger\ket{n_1,n_2,\cdots,n_k}=0\text{  if }n_i=1 \\
\end{aligned}
\]
\[
\begin{aligned}
b&_i\ket{n_1,n_2,\cdots,n_k}=\ket{n_1,n_2,\cdots,n_i-1,\cdots,n_L}=\ket{n_1,n_2,\cdots,0,n_{i+1},\cdots,n_L}\text{if }n_i=1 \\
&b_i\ket{n_1,n_2,\cdots,n_k}=0\text{  if }n_i=0 \\   
\end{aligned}
\]
$[b_i,b_j]=0 \text{for any i,j} \text{because when i=j} (b_ib_j-b_jb_i)\ket{n_1,n_2,\cdots,n_L}=0\text{if}n_i=0$
$[b_i,b_j]=0\text{if}n_i=1\text{because}b_ib_j\ket{n_1,n_2,\cdots,n_L}=\ket{n_1,n_2,\cdots,n_L} \text{and}b_jb_i\ket{n_1,n_2,\cdots,n_L}=\ket{n_1,n_2,\cdots,n_L}$
when $i\neq j$, $(b_ib_j-b_jb_i)\ket{n_1,n_2,\cdots,n_L}=0$ because $b_ib_j\ket{n_1,n_2,\cdots,n_L}$ and $b_jb_i\ket{n_1,n_2,\cdots,n_L}$ are the same state.
$[b_i,b_j^\dagger]=\delta_{ij}$

\textbf{case 1: i=j}
\[[b_i,b_i^\dagger]=b_ib_i^\dagger-b_i^\dagger b_i\]
if $n_i=0$, we have:
\[\begin{aligned}
&b_ib_i^\dagger\ket{n_1,n_2,\cdots,n_k} = b_i\ket{n_1,n_2,\cdots,1,n_{i+1},\cdots,n_L} \\
&=\ket{n_1,n_2,\cdots,0,n_{i+1},\cdots,n_L}=\ket{n_1,n_2,\cdots,n_k} \\
\end{aligned}
\]
$b_i^\dagger b_i\ket{n_1,n_2,\cdots,n_k}=b_i^\dagger 0=0$

\text{case 2: i$\neq$j}
\[[b_i,b_j^\dagger]=b_ib_j^\dagger-b_j^\dagger b_i
=0\text{because when}n_i=0\text{or}n_j=0,b_ib_j^\dagger\ket{n_1,n_2,\cdots,n_L}=0\text{and}b_j^\dagger b_i\ket{n_1,n_2,\cdots,n_L}=0\]
when $n_i=1$ or $n_j=1$, $b_ib_j^\dagger\ket{n_1,n_2,\cdots,n_L}$ and $b_j^\dagger b_i\ket{n_1,n_2,\cdots,n_L}$ are the same state.
only when $n_i=1$ and $n_j=0$, we have:
\[
\begin{aligned}
&b_ib_j^\dagger\ket{n_1,n_2,\cdots,1,\cdots,0,\cdots,n_L} = b_i\ket{n_1,n_2,\cdots,1,\cdots,1,\cdots,n_L} \\
&=\ket{n_1,n_2,\cdots,0,\cdots,1,\cdots,n_L}=b_j^\dagger b_i \\
\end{aligned}
\]

\textbf{(2)}
$\sigma_i^{\pm}=(\sigma_i^x\pm\sigma_i^y\cdot i)/2$if we call
\textbf{case 1}
$[\sigma_i^-,\sigma_j^-]=0$
when $i=j$, we have:
\[
[\sigma_i^-,\sigma_i^-]=\sigma_i^-\sigma_i^--\sigma_i^-\sigma_i^-=0 
\textbf{(3)}
\]
when $i\neq j$, we have:
\textbf{(4)}
how should the Pauli operators $\sigma_i^+ $ and $\sigma_i^z $ can be written in terms of the  $f$ operators? 
we have the definition that:
\[
f_i^\dagger = \left( \prod_{j<i} \sigma_j^z \right) \sigma_i^+
\]

so from $\braket{\overline{n_{1'}n_{2'}\cdot n_{i'}}|f_i^\dagger|\overline{n_1n_2n_3\cdot n_i}}$=$\braket{\overline{n_{1}n_{2}\cdot n_{i}}|f_i|\overline{n_{1'}n_{2'}\cdot n_{i'}}}^*$
we can get the relation between $f_i$ and $\sigma_i^+$, $\sigma_i^z$ as follows:
\[
f_i = \left( \prod_{j<i} \sigma_j^z \right) \sigma_i^-
\]

then try to represent $\sigma_i^z$ and $\sigma_i^+$ in terms of $f_i$:
we use the matrix representation of the Pauli operators:
\[
\sigma_i^z = \begin{pmatrix}
1 & 0 \\
0 & -1
\end{pmatrix}, \quad \sigma_i^+ = \begin{pmatrix}
0 & 1 \\
0 & 0
\end{pmatrix}, \quad \sigma_i^- = \begin{pmatrix}
0 & 0 \\
1 & 0
\end{pmatrix}
\]
$\sigma_i^z  $can be expressed as:$  2\sigma_i^+\sigma_i^- -I = 2f_i^{\dagger} f_i-I$


like wise:
\[
\sigma_i^+ = f_i^{\dagger} \left( \prod_{j<i} \sigma_j^z \right)^{-1}
\]
\[
= f_i^{\dagger} \left( \prod_{j<i} (2f_j^{\dagger} f_j - I) \right)
\]




\end{document}

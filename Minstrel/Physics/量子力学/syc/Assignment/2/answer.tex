\documentclass[12pt]{article}
\usepackage{amsmath, amssymb, physics, geometry}
\usepackage{braket}
\usepackage{braket}
\usepackage{ctex}
\geometry{a4paper, margin=1in}
\usepackage{hyperref}

\title{The Answer of Assignment 2}
\author{WEI SHUANG}
\date{31/8/2025}

\begin{document}
\maketitle

\begin{center}
    \textbf{Problem 1 Solution}
\end{center}

\textbf{(1)}
根据线性空间的同构理论,任何维数相同的线性空间都是同构的。因此  
可知$\sum\limits_{i,j=0}^{1}c_{ij}\ket{i,j}$与$\{(x_1,x_2,x_3,x_4)\},x\in C$同构。即$v_1=(c_{00},c_{01},c_{02},c_{03})^*$,$v_2=(d_{00},d_{01},d_{10},d_{11})^*$




\textbf{(2)}
$\braket{\psi_1|\psi_2}=v_1^\dagger v_2$成立,这是因为

\textbf{(3)}

\textbf{(4)}
\begin{center}
    \textbf{Problem 2 Solution}
\end{center}

\textbf{(1)}

\textbf{(2)}

\textbf{(3)}

\textbf{(4)}

\begin{center}
    \textbf{Problem 3 Solution}
\end{center}

\textbf{(1)}

\textbf{(2)}

\textbf{(3)}

\textbf{(4)}
how should the Pauli operators $\sigma_i^+$ and $\sigma_i^z be written in terms of the  $f$ operators? $
\end{document}

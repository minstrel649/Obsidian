\documentclass[12pt]{article}
\usepackage{amsmath, amssymb, physics, geometry}
\usepackage{braket}
\usepackage{braket}
\usepackage{ctex}
\geometry{a4paper, margin=1in}
\usepackage{hyperref}

\title{The Answer of Assignment 2}
\author{WEI SHUANG}
\date{31/8/2025}

\begin{document}
\maketitle

\begin{center}
    \textbf{Problem 1 Solution}
\end{center}

\textbf{(1)}
根据线性空间的同构理论,任何维数相同的线性空间都是同构的。因此  
可知$\sum\limits_{i,j=0}^{1}c_{ij}\ket{i,j}$与$\{(x_1,x_2,x_3,x_4)\},x\in C$同构。即$v_1=(c_{00},c_{01},c_{02},c_{03})^*$,$v_2=(d_{00},d_{01},d_{10},d_{11})^*$




\textbf{(2)}
$\braket{\psi_1|\psi_2}=v_1^\dagger v_2$成立,这是因为
$\braket{\psi_1|\psi_2}=(\sum_{i,j=0}^1c_{i,j}^*\bra{ij})(\sum{i,,j=0}^1c_{i,j}^*\ket{ij})=v_1^\dagger v_2$
\textbf{(3)}

\textbf{(4)}
\begin{center}
    \textbf{Problem 2 Solution}
\end{center}

\textbf{(1)}

\textbf{(2)}

\textbf{(3)}

\textbf{(4)}

\begin{center}
    \textbf{Problem 3 Solution}
\end{center}

\textbf{(1)}

\textbf{(2)}

\textbf{(3)}

\textbf{(4)}
how should the Pauli operators $\sigma_i^+ $ and $\sigma_i^z $ can be written in terms of the  $f$ operators? 
we have the definition that:\[
f_i^\dagger = \left( \prod_{j<i} \sigma_j^z \right) \sigma_i^+
\]

so from $\braket{\overline{n_{1'}n_{2'}\cdot n_{i'}}|f_i^\dagger|\overline{n_1n_2n_3\cdot n_i}}$=$\braket{\overline{n_{1}n_{2}\cdot n_{i}}|f_i|\overline{n_{1'}n_{2'}\cdot n_{i'}}}^*$
we can get the relation between $f_i$ and $\sigma_i^+$, $\sigma_i^z$ as follows:
\[
f_i = \left( \prod_{j<i} \sigma_j^z \right) \sigma_i^-
\]

then try to represent $\sigma_i^z$ and $\sigma_i^+$ in terms of $f_i$:
we use the matrix representation of the Pauli operators:
\[
\sigma_i^z = \begin{pmatrix}
1 & 0 \\
0 & -1
\end{pmatrix}, \quad \sigma_i^+ = \begin{pmatrix}
0 & 1 \\
0 & 0
\end{pmatrix}, \quad \sigma_i^- = \begin{pmatrix}
0 & 0 \\
1 & 0
\end{pmatrix}
\]
$\sigma_i^z  $can be expressed as:$  2\sigma_i^+\sigma_i^- -I = 2f_i^{\dagger} f_i-I$


like wise:
\[
\sigma_i^+ = f_i^{\dagger} \left( \prod_{j<i} \sigma_j^z \right)^{-1}
\]
\[
= f_i^{\dagger} \left( \prod_{j<i} (2f_j^{\dagger} f_j - I) \right)
\]




\end{document}
